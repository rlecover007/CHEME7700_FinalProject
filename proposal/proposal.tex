\documentclass{article}
\usepackage[utf8]{inputenc}
\usepackage{fancyhdr}
\usepackage{graphicx}
\usepackage{float}
\usepackage{pdflscape}
\usepackage[margin=1.5in]{geometry}
 
\pagestyle{fancy}
\fancyhf{}
\rhead{Rachel LeCover}
\lhead{Project Proposal}
\rfoot{Page \thepage}
 
\begin{document}
\section*{Topic}
Platelets play a key role in coagulation. Not only do they contribute to the structure of clots, the proteins they release accelerate the coagulation cascade. Furthermore, when activated by the presence of thrombin, a key protease in the coagulation cascade, they express negatively charged lipids on their surfaces, further promoting coagulation.
\section*{Current Literature}
Thomas et al have developed a model for the metabolic network of a platelet, as detailed in \cite{thomas2014network}. This network contains 1,008 reactions, with 739 compartment specific metabolites and 225 proteins. Furthermore, it encodes which reactions are controlled by which genes, allowing for the simulation of knock outs. 
\section*{Extension of Model}
Thomas et al investigated the effect on aspirin on the system, but did not develop a dynamic model of the system. I propose to use this model of platelet metabolism to dynamically model the state of the platelet using dFBA. Once my dynamic model is constructed, I can use it to simulate how platelets would function should certain genes be knocked out. Although other kinetic models of platelets exist  \cite{ashby1989model}, I believe that no one has ever constructed a dFBA model of platelet metabolism. 

\bibliographystyle{unsrt}
\bibliography{propsal}
\end{document}